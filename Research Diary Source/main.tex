%%% Research Diary - Entry
%%% 
\documentclass[11pt,letterpaper]{article}

\newcommand{\workingDate}{\textsc{2018 $|$ September $|$ 01}}
\newcommand{\userName}{Pratham Gandhi, Samuel Schuur}
\newcommand{\institution}{}
\usepackage{researchdiary_png}
% To add your univeristy logo to the upper right, simply
% upload a file named "logo.png" using the files menu above.
\usepackage{hyperref}

\begin{document}

\title{Constellation Project Research Diary}

{\Huge July 18}\\[5mm]

\textit{Summary: Our materials have not arrived yet. We looked into other systems or algorithms which are currently being used to achieve similar outcomes as Constellation is proposed to have.}

\section{Comparable Systems}

\subsection{Structure-From-Motion Pipeline}
\subsubsection{Epipolar Geometry and Essential Matrices}
\begin{itemize}
\item We have two views of a scene, taken from different viewpoints
\item We see an image point p in one image, which is the projection of a 3D point
\item Given $p_0$ in the first image, where can the corresponding point $p_1$ in the second image be?
\item the vector from Camera0 to p0 ($\vec{C_0p_0}$) is coplanar with $\vec{C_0C_1}$ and $\vec{C_1p_1}$, which can be expressed mathematically if you take the cross product of two of them, in this case $\vec{C_0C_1}$ and $\vec{C_1p_1}$, which should be perpendicular to the third vector, the dot product of the third vector with that should equal 0 ($\vec{C_0p_0}\cdot(\vec{C_0C_1}\times\vec{C_1p_1})=0$), because p0 lies in that same plane
\item We are going to think of p0 not as a point, but now as a 3D direction vector ($\begin{pmatrix}x_0 \\ y_0\\ 1\end{pmatrix}$) - x0 and y0 are projected points on the image plane, and it's a normalized image with focal length=1
\item If I want to express p1 in camera0 coordinates I need to rotate it by the rotation matrix between C1 and C0
\begin{itemize}
\item $^{C_0}_{C_1}Rp_1$
\end{itemize}
\item use that equation for lots of known points to determine the 3d direction vectors of each of the points, as shown in \href{https://www.youtube.com/watch?v=kfN76APa4HE}{this} lecture
\end{itemize}
\subsubsection{Observations}
\begin{itemize}
\item Results can be unstable, due to poor numerical conditioning
\begin{itemize}
\item A little bit of image noise can cause a large error in the resulting essential matrix
\end{itemize}
\item We can improve the results by:
\end{itemize}
\subsection{Stereo Vision}

\end{document}