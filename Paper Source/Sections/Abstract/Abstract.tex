A system to extract 3D information about a space was created that could be trained simply in a matter of minutes. The system, nicknamed "Constellation", was designed to be implemented in situations requiring rapidly updating computations and environmental awareness, to run on relatively inexpensive hardware, and to perform its tasks with limited human intervention. To achieve this, Constellation uses a combination of hardware and software components, focused around a set of artificial neural networks to extract features and a physical grid of points refracted from a single laser. A Python 3.6 implementation of the system was tested on two machines in a variety of environments, and produced an average of \textit{x} times faster environment generation as other comparable algorithms or approaches to similar problems.