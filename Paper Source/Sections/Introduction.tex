The current most popular approach to creating a three dimensional informational estimate from a two dimensional image is triangulation, or binocular, stereo vision approach. This approach take several images of the same scene from different locations and angles in order to generate a spatial awareness. Computer stereo vision, however, is difficult to implement in practical applications in our increasingly evolving digital world, due to the limiting factor that they require multiple images, the acquisition of which is sometimes impractical. Additionally, they can sometimes be quite slow, in that they need to first identify common features in the different photos, find out the positional relation of those features, and then only begin to start seeing the greater picture of the full space. The purpose of Constellation was to solve both of these problems and create a flexible system which can adapt and continue to be functional in many environments.

We chose to approach this problem first through the somewhat traditional technique of using artificial neural networks, mathematical functions modeled after nature which specialize in taking in large amounts of input to produce outputs, to identify features in a two dimensional image. Neural networks are are perfect for navigating the fuzzy problems of feature extraction from images. Where our system differs is in its use of a grid of refracted laser points into the scene to judge the distances and orientations of various objects in relation to the camera. This allows superior environment generation in a shorter amount of time, due to the omission of the multiple-image stereo aspect. This approach and its implemented system was named Constellation for the star-like appearance of the laser dot grid it relies on.