\subsubsection{Why Neural Networks?}
The central problem detecting various features within an image is one of classification. There exist several different approaches or algorithms for classification, including decision trees, naive, bayes, and KNN. However, only two classification models stood out for our particular application of object recognition; support vector machines (SVMs), and neural networks. SVMs operate by determining an optimal classification line, which seperates training data of two different types into two distinct sections, and performs classification by determining which side of that line a set of input data falls. While SVMs have a higher accuracy in general, and are more tollerant to redundant and irrelevant attributes, they require on average three times more training samples to accurately classify features than neural networks, and still perform evenly with neural networks with regards to the speed of classification, speed of learning, and, tolerance to highly interdependent attributes. This system's applications require rapidly updating environments, which thus require rapid computations and rapid training. These reasons make SVMs impractical for use in Constellation, which means the best choice for Constellation's object detection and classification approach was neural networks, which could easily be pretrained and then adjusted during operation.
\subsubsection{Differentiating Objects and Laser Dots Using Efficient Universal Algorithms}